\documentclass{beamer}
\usetheme{Madrid}
\usecolortheme{default}
\usepackage{graphicx} % Required for inserting images
\graphicspath{ {./images/} }
\usepackage{geometry}
\usepackage[T1]{fontenc}
\usepackage{fbb}
\usepackage{amsthm}
\usepackage[libertine]{newtxmath} 
\usepackage[italic]{mathastext}
\MTsetmathskips{f}{5mu}{1mu} 
\usepackage[utf8]{inputenc}
\usepackage{textcomp}
\usepackage{graphicx}
\usepackage{listings}
\usepackage{caption}
\usepackage{subcaption}
\usepackage{lipsum}
\usepackage{setspace}
\usepackage{chngcntr}
\usepackage{fancyhdr}
\usepackage{listings}
\usepackage{multirow}
\usepackage{scalerel}
\DeclareMathOperator*{\bigplus}{\scalerel*{+}{\sum}}
\usepackage{tikz-cd}
\usetikzlibrary{patterns.meta}
\usepackage{hyperref}


\title{
    The Hypercubic Manifold\\
    in\\
    Homotopy Type Theory}
\date{04/09/2023}
\author{Dylan Laird}

\begin{document}

\frame{\titlepage}

\begin{frame}
    \frametitle{Table of contents}
    \tableofcontents
\end{frame}
\AtBeginSection[]
{
  \begin{frame}
    \frametitle{Table of contents}
    \tableofcontents[currentsection]
  \end{frame}
}
\section{A brief overview of HoTT/cubical type theory}
    \subsection{Informal Type Theory}
        \begin{frame}
            \frametitle{Type Theory}
            We manipulate types $A,B$ and terms $a : A$ of some certain type. \\
            \pause
            \begin{exampleblock}{The inductive type of natural integers}
                \begin{itemize}
                    \item $0 : \mathbb N$
                    \item $\mathrm{suc}: \mathbb N \rightarrow \mathbb N$
                \end{itemize}
            \end{exampleblock}
            \pause
            \begin{block}{Equality}
                We distinguish between \textbf{definitional equality} $\equiv$ used when defining objects and \textbf{propositional equality} which is a type theoretical concept.
            \end{block}
            \pause
            \begin{exampleblock}{Propositional Equality}
                Given a type $A$ and terms $a,b : A$ there is a type $a=_A b$. We say that elements $a$ and $b$ are (propositionally) equal when $a=_A b$ is inhabited.
            \end{exampleblock}
        \end{frame}

        \begin{frame}
            \frametitle{Some constructions on types}
            \begin{exampleblock}{Function types}
                Given types $A,B$ one has a type $A \rightarrow B$ of functions from $A$ to $B$.
            \end{exampleblock}
            \pause
            \begin{exampleblock}{Product type}
                Given types $A,B$ one has a product type $A \times B$.
            \end{exampleblock}
            \pause
            \begin{exampleblock}{Coproduct type}
                Give types $A,B$ one has a coproduct type $A+B$ given by the constructors :
                \begin{itemize}
                    \item inl : $A \rightarrow A+B$
                    \item inr : $B \rightarrow A+B$
                  \end{itemize}
            \end{exampleblock}
        \end{frame}

        \begin{frame}
            \frametitle{Manipulating types}
            \begin{block}{Introduction rules}
                They encapsulate how to build an element of a certain type.
                \begin{itemize}
                    \item To build an element of $ f: A \rightarrow B$ one needs an expression $\phi(x)$ such that $a : A \vdash \phi(a) : B$ and to set $f :\equiv \lambda x. \phi(x)$
                    \item To build an element of $A \times B$ one needs to take elements $a : A$ and $b : B$ to form $(a,b) : A \times B$
                \end{itemize}
            \end{block}
            \pause
            \begin{block}{Induction principles}
                They encapsulate how to build \textbf{dependent} functions from a source type $A$.
                \begin{itemize}
                    \item To build a function of type $f : \prod_{x : A\times B} P(x)$ one only needs to give its value on pairs $(a,b)$. 
                \end{itemize}
            \end{block}
        \end{frame}

        \begin{frame}
            \frametitle{A working example : product types (1)}
            \begin{exampleblock}{Projections}
                We can define projections $\mathrm{pr}_1 : A\times B \rightarrow A$ and $\mathrm{pr}_2 : A\times B \rightarrow B$ by the \textbf{induction} principle for product types by setting $\mathrm{pr}_1((a,b)) :\equiv a$ and $\mathrm{pr}_2((a,b)) :\equiv b$.
            \end{exampleblock}
            \pause
            \begin{exampleblock}{Propositional uniqueness for product types}
                We have a type theoretic statement that expresses that the type $A \times B$ is the type of "pairs of elements of $A$ and $B$" : 
                $$ \mathrm{uniq}_{A\times B} : \prod_{x : A \times B} \big(x =_{A \times B} (\mathrm{pr}_1(x),\mathrm{pr}_2(x))\big) $$
            \end{exampleblock}
        \end{frame}
        \begin{frame}
            \frametitle{A working example : product types (2)}
            Sketch of proof : 
            \begin{itemize}
                \item By induction, we only need to build an element : 
                $$\mathrm{uniq}_{A \times B} ((a,b)) : (a,b)=(\mathrm{pr}_1(a,b),\mathrm{pr}_2(a,b))$$
                \pause
                \item By the definition of the projections, the goal type reduces to: 
                $$(a,b)=_{A\times B} (a,b)$$
                \pause
                \item We now have an element $\mathrm{refl} : (a,b)=_{A\times B} (a,b)$ \qed
            \end{itemize}
            
        
        \end{frame}
\end{document}